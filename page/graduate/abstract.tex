\cleardoublepage
\chapternonum{摘要}
四足机械臂兼具足式机器人的全地形通过能力与机械臂的灵巧操作能力,在灾后救援、工业巡检及家庭服务等非结构化场景中具有广阔的应用前景。然而,该系统具有高维冗余、强动力学耦合及异构执行器特性,如何实现全系统的高效全身协调控制,尤其是在高动态干扰下的鲁棒操作,是当前机器人领域的难点。深度强化学习为解决此类端到端控制问题提供了新范式,但在处理移动与操作这两个任务目标时,往往面临探索效率低、易陷入局部最优以及仿真到现实迁移困难等挑战。针对上述问题,本研究提出了一套完整的四足机械臂全身协同控制框架,主要研究内容与贡献如下:

第一,构建了基于非对称 Actor-Critic 架构的全身控制学习框架。将控制问题建模为部分可观测马尔可夫决策过程,针对真实环境中感知信息不全的挑战,策略网络采用 LSTM 单元从含噪的本体观测中隐式推断系统状态;价值网络则融合了地形特征与特权状态信息以提升价值评估准确性。针对训练过程,设计了包含动力学参数随机化与地形难度自适应的双重课程学习策略,有效提升了算法在高维动作空间中的收敛速度。

第二,提出了物理可行域引导的奖励重塑方法。针对强化学习在训练时因过度关注某一任务而导致协同行为探索停滞的问题,本研究引入了基于迭代逆运动学的实时求解器,形式化定义了“物理可行状态”。该方法将四足机器人与机械臂复杂的动力学约束解耦,通过高斯核函数与最小激励机制,引导策略主动探索能够扩展机械臂操作空间的躯干姿态。同时,设计了基于投影的速度误差分解与基于关键点的隐式位姿度量方法,解决了多目标优化中的量纲统一难题。实验表明,PFG 机制成功引导策略突破了局部最优,将机械臂的有效工作空间显著扩展了 $34\%$。

第三,建立了一套高保真的 Sim-to-Real 迁移策略。针对四足底盘与机械臂执行器在刚度配置上的显著异构性,建立了包含双队列延迟注入与宽限幅力矩限幅奖励的高保真执行器模型,解决了高增益关节在训练中的奖励饱和问题。配合基于监督学习的神经网络速度估计器与传感器噪声模拟,有效弥合了仿真与现实的动力学鸿沟。

第四,搭建了基于 X20-Z1 的物理样机平台并完成了多维度的实物验证。实验结果表明,所提策略具备极强的鲁棒性与泛化能力:在全身运动学协同实验中,机器人展现了自发的姿态自适应行为;在极具挑战的模仿体操运动员挥舞彩带任务中,系统成功抑制了机械臂以 $2\pi\,\text{rad/s}$ 高速运动产生的内源性惯性扰动,保持了四足机器人运动的稳定;此外,在非预设的机械臂失能故障实验中,策略展现了零样本泛化能力,成功实现了安全容错控制;最后,通过多场景的操作实验,展示了所提方法的有效性与多场景应用。

\vspace{1em}
\noindent \textbf{关键词:} 四足机械臂;全身协同控制;深度强化学习;物理可行域引导

\cleardoublepage
\chapternonum{Abstract}
